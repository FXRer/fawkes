%
% $Id$
%
% Copyright (c)  2007  Tim Niemueller, AllemaniACs RoboCup Team.
%
% Created: Wed Jan 24 2007 15:31:18

\section{Development}

\frame{\tableofcontents[currentsection,hideothersubsections]}

\subsection{Principles}
\begin{frame}
  \frametitle{Fawkes Development Principles}
  \begin{itemize}[<+>]
  \item Document all your code immediately
  \item Fix bugs before implementing new stuff
  \item Exploit all available tools, use gdb and valgrind
  \item Use what's there, do not re-invent the wheel
  %\item Review code of others and make them review your code
  \end{itemize}
\end{frame}

\subsection{Improvements}
\begin{frame}
  \frametitle{Improvements to Development Tools}
  \begin{itemize}
  \item<1> Subversion for version control
    \begin{itemize}
    \item Finally refactoring is fun (move cmd)
    \item Blame command to see originator
    \item Offline operations for status, diffs etc.
    \end{itemize}
  \item<2> Trac
    \begin{itemize}
    \item Source code browser
    \item Ticket management for bugs and features
    \item Timeline and Roadmap
    \item Access to documentation, API reference and wiki
    %\item Centralized user mangement independent of Unix accounts
    \end{itemize}
  \end{itemize}
\end{frame}

\subsection{Implementing a Plugin}
\begin{frame}
  \frametitle{Creating the Threads}
  \begin{itemize}
  \item Derivative of Thread
  \item If needed use WAITFORWAKEUP mode (thread will wait after every loop for a
    wake-up call, needed for BlockedTimingAspect)
  \item Do all initialisation in the constructor
  \item Implement \texttt{loop()} to do what you need to do
  \item Add any aspect that you need by deriving its aspect class
  \item If threads need synchronisation among each other pass the needed constructs
    to the constructor (consider a synchronized shared data object)
  \end{itemize}
\end{frame}

\begin{frame}
  \frametitle{Creating a Plugin}
  \begin{itemize}
  \item Derivative of Plugin
  \item Implement plugin's threads
  \item Implement \texttt{threads()} to return a list of instantiated
    threads, take care of inter-thread synchronisation details here
  \item Implement \texttt{plugin$\_$factory()} and \texttt{plugin$\_$destroy()}
  \end{itemize}
  \begin{block}<2>{First steps}
    Use \texttt{src/plugins/example} as a template. It contains a basic plugin
    that will run a few simple threads.
  \end{block}
\end{frame}

\begin{frame}
  \frametitle{Creating an Interface}
  \begin{itemize}
  \item Write XML template in \texttt{src/interfaces}
  \item \texttt{make}
  \item This will build \texttt{.h}/\texttt{.cpp} file and compile
  \item Use it
  \item Documentation yet to be written
  \end{itemize}
\end{frame}

\subsection{Adding an Aspect}
\begin{frame}
  \frametitle{Creating an Aspect}
  \begin{itemize}
  \item Plain class, may not derive anything
  \item May use any library, avoid big fat external dependencies
  \item May not have pure virtual functions
  \item May have special constructor
  \item May have initialization routine, name specific to avoid name clashes
    (not \texttt{init()} but \texttt{MyAspect::initMyAspect()})
  \item Make AspectInitializer know how to initialize the aspect and to detect
    any problems to meet guarantees
  \item Document extensively
  \end{itemize}
\end{frame}

\subsection{Running Fawkes}
\begin{frame}
  \frametitle{Fawkes Tools}
  \begin{itemize}
  \item \texttt{config}: Configuration editing over the network
  \item \texttt{plugin}: Load and unload plugins
  \item \texttt{interface$\_$generator}: Transform BB interface XML templates into code
  \item use \texttt{-H} argument for a usage message (file a bug if missing!)
  \end{itemize}
\end{frame}

\begin{frame}
  \frametitle{Running Fawkes}
  \begin{center}
    \texttt{./fawkes}
  \end{center}
\end{frame}


%%% Local Variables: 
%%% mode: latex
%%% TeX-master: "fawkes-glance"
%%% End: 
