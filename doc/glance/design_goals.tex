%
% $Id$
%
% Copyright (c)  2007  Tim Niemueller, AllemaniACs RoboCup Team.
%
% Created: Wed Jan 24 2007 15:30:24

\section{Design Goals}

\frame{\tableofcontents[currentsection,hideothersubsections]}

% \begin{block}{}
%   \begin{quote}
%     Good judgement comes from experience. Experience comes
%     from bad judgement.(G. Warren Nutter)
%   \end{quote}
% \end{block}

\subsection{Masterplan}
\begin{frame}
  \frametitle{What to Keep}
  \begin{itemize}
  \item Keep modular structure
  \item Keep unified information storage (similar to BlackBoard)
  \item Provide simple templates for new modules
  \item Provide the well-known tools (RCCC, vis$\_$bb, FireStation)
  \end{itemize}
\end{frame}

\begin{frame}
  \frametitle{What to Avoid}
  \begin{itemize}
  \item Avoid polling, events and blocked waiting instead
  \item Avoid dependencies where possible, have few and document them well
  \item Avoid code duplication and more approaches where useless
    (different visions, localizations etc. may be OK, having two thread implementations
    is useless)
  \item Avoid everything that makes debugging hard or impossible
  \end{itemize}
\end{frame}

\begin{frame}
  \frametitle{What to Add and Improve}
  \begin{itemize}
  \item Source code management
  \item Enforce documentation
  \item Sleek and fast build system
  \item Easy modular software structure
  \item Mutual exclusions on information storage
  \item Throw away stuff not needed any more, it's still in SVN!
  \item Make it debuggable and do it! (gdb, valgrind, QA apps, unit tests)
  \item Use exceptions for good error handling and better readability
  \end{itemize}
\end{frame}

\begin{frame}
  \frametitle{Guarantees}
  \begin{itemize}
  \item Guarantees ensure certain conditions, behavior and functionality of
    the software stack
  \item<2-> Have guarantees!
  \item<2-> Guarantees minimize error handling in upper levels
  \item<2-> Failures in guaranteed components are a bug. No more discussion about that.
  %\item With exception handling errors are well defined and can be handled
  \item<2-> Not met guarantees are crash points by design
  \end{itemize}
  \begin{block}<3>{}
    \centering
    Guarantees are needed to keep the code simple, to have well defined
    software interfaces and to minimize the risk of errors.
  \end{block}
\end{frame}

\begin{frame}
  \frametitle{Needed Guarantees (known so far)}
  \begin{itemize}[<+->]
  \item \emph{Initialization:} either a component/thread/aspect is successfully
    initialized or never started
  \item \emph{Dependencies:} Either all requirements are met or a component cannot run
    (and the problem is detected!)
  \item \emph{Concurrency:} There must be mutually exclusive access to critical components
    like data storage (single writer)
  \item \emph{Timing:} Guarantee a defined and documented call chain per loop
  \item \emph{Time Source:} Guarantee that all components use the exact same time source
  \end{itemize}
\end{frame}

\subsection{Summary}
\begin{frame}
  \frametitle{Summary}
  \begin{itemize}[<+->]
  \item Modular
  \item Unified information storage
  \item No polling, events
  \item Central timing and time source
  \item Debuggable
  \item Guarantees
  \item Minimize dependencies
  \end{itemize}
  \begin{block}<8>{}
    \centering
    One to rule them all: only one dynamic application
  \end{block}
\end{frame}

%%% Local Variables: 
%%% mode: latex
%%% TeX-master: "fawkes-glance"
%%% End: 
